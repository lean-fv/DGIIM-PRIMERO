% !TEX TS-program = XeLaTeX
% !TEX encoding = UTF-8 Unicode



\documentclass[10pt,xcolor=svgnames]{beamer}
\usepackage{metalogo}
\setlogokern{La}{-0.15em}
\setlogokern{aT}{-0.07em}
\setlogokern{eX}{-0.07em}



\usepackage{fontawesome}



\usepackage[utf8]{inputenc}       
\usepackage[T1]{fontenc}

\usepackage[spanish]{babel}

% \deactivatequoting
% \let\layoutspanish\relax



\usepackage{tikz,pgf}%,pgfarrows,pgfnodes,pgfautomata,pgfheaps}


\usepackage{booktabs}


\usetheme[sectionpage=progressbar,numbering=none,titleformat frame=smallcaps]{metropolis}



\setbeamertemplate{navigation symbols}{}



\definecolor{dblue}{rgb}{0.23,0.4,0.7}
\definecolor{azulon}{rgb}{0,0,0.44}
\definecolor{naranjon}{rgb}{.84,.418,0}
\definecolor{rojoclaro}{rgb}{.6,.2,0}
\definecolor{verdepaquete}{rgb}{.0,.4,.2}

\definecolor{codegreen}{rgb}{0,0.6,0}
\definecolor{codegray}{rgb}{0.5,0.5,0.5}
\definecolor{codepurple}{rgb}{0.58,0,0.82}
\definecolor{backcolour}{rgb}{0.95,0.95,0.92}

\usepackage{multicol}


\usepackage[random]{blindtext}
\usepackage{colortbl}
\usepackage{graphicx}


%% Cajas
\usepackage[listings]{tcolorbox}




\usepackage{keystroke}


\usepackage{listings}
\lstset{
	language=[latex]tex,
	backgroundcolor=\color{backcolour},   
    commentstyle=\color{codegreen},
    keywordstyle=\color{magenta},
    numberstyle=\tiny\color{codegray},
    stringstyle=\color{codepurple},
    basicstyle=\ttfamily,%\footnotesize,
    breakatwhitespace=false,         
    breaklines=true,  
    firstnumber=1,
%    captionpos=b,                    
    keepspaces=true,                 
    numbers=left,                    
    numbersep=5pt,                  
    showspaces=false,                
    showstringspaces=false,
    showtabs=false,                  
    tabsize=2}

% minimizar fragmentado de listados
\lstnewenvironment{listing}[1][]
   {\lstset{#1}\pagebreak[0]}{\pagebreak[0]}

\lstdefinestyle{consola}
   {basicstyle=\scriptsize\bf\ttfamily}







\hypersetup{pdftitle={Taller de LaTeX},
  pdfsubject={Curso de LaTeX}
  pdfauthor={Orientamat}
  pdfkeywords={LaTeX, Orientamat},
  pdfpagemode={FullScreen},
%  colorlinks,
%  linkcolor=Orange,
  }




\title{Taller de \LaTeX}

\subtitle{Instalación, personalización y primeros pasos}

\author{Orientamat}

\institute{Universidad de Granada}

\date[2017]{24 de marzo de 2017}

%\titlegraphic{\includegraphics[scale=.25]{UGR-grlogo.png}}

\begin{document}


\maketitle


\begin{frame}
\frametitle{Estructura del Curso}
  \setbeamertemplate{section in toc}[sections numbered]
  \tableofcontents[hideallsubsections]
\end{frame}



\section{Instalación}

\begin{frame}
\frametitle{Distribuciones}
\begin{itemize}
\addtolength{\itemsep}{0.6\baselineskip}
\item \LaTeX{} está disponible en la mayoría de las plataformas usuales



\item La \emph{distribuciones} más populares son
	\medskip

	\begin{itemize}
	\addtolength{\itemsep}{0.5\baselineskip}
	\item MiK\TeX{} (\faWindows)
	\item Mac\TeX{} (\faApple)
	\item \TeX Live (\faApple, \faLinux, \faWindows)
	\end{itemize}

\item Todas las distribuciones están basadas en el material disponible en CTAN.	
\end{itemize}
\end{frame}


\begin{frame}
\frametitle{Instalación}

Es importante que tengamos instalado algún visor de archivos PDF.

\begin{block}{En Windows \faWindows}
    \begin{itemize}
    %\setlength\itemsep{0.2\baselineskip}
        \item Vamos a instalar la distribución MiK\TeX
        \item Usaremos una variante de esta, \alert{Pro\TeX{t}}, que tiene incluidas algunos programas adicionales como TeXstudio o Ghostscript. 

        \centerline{\small \url{https://tug.org/protext/}}

    \end{itemize}
\end{block}


\begin{block}{En macOS \faApple}
    \begin{itemize}
        \item Usar \alert{Mac\TeX{}}  \quad {\small \url{https://tug.org/mactex/}}
    \end{itemize}
\end{block}

\begin{block}{En Linux \faLinux}
    \begin{itemize}
        \item Está disponible en los repositorios de las distribuciones
    \end{itemize}
\end{block}
\end{frame}


\begin{frame}
\frametitle{Editores}

El programa (editor) que usemos para escribir un documento es independiente de \LaTeX{} aunque existen algunos editores mejor adaptados a su uso que incluyen atajos para algunas acciones usuales.

\medskip

Los más comunes son
\begin{description}
	\addtolength{\itemsep}{0.5\baselineskip}
	\item[\faWindows] TeXstudio, Texworks, Texniccenter, Texmaker (varias plataformas), WinEdt (shareware), Led,...
	\item[\faApple] TeXShop, Texmaker, Texworks, scite,...
	\item[\faLinux] Kile, Texworks, emacs, vim,...
\end{description}
\end{frame}

\section{Generalidades}

\subsection{¿Qué es \LaTeX?}

\begin{frame}
\frametitle{¿Qué es?}
	\begin{block}{¿Qué es \TeX?}
		\begin{itemize}
			\item \TeX{} es un programa destinado a la composición de documentos que contienen texto y fórmulas matemáticas con calidad de imprenta creado por Donald Knuth en 1978
			\item NO es un editor de texto sino un procesador de macros y lenguaje de programación
		\end{itemize}
	\end{block}
	\begin{block}{¿Y \LaTeX?}
		\begin{itemize}
			\item \LaTeX{} es un conjunto de macros para \TeX{}  debido originalmente a Leslie Lamport para facilitar el uso de \TeX.
			\item La Sociedad Matemática Americana añade sus estándares a \LaTeX{}: nace AMS-\LaTeX
		\end{itemize}
	\end{block}

	Usaremos el término \alert{\LaTeX{}} para referirnos a \TeX{} + \LaTeX{} + mejoras sucesivas
\end{frame}
 



\begin{frame}
\frametitle{Características de \LaTeX}
	\begin{description}[Transportabiilidad]
	\addtolength\itemsep{\fill}
		\item[Transportable] los ficheros .tex sólo contienen texto y son de pequeño tamaño 
		\item[WYSIWYM] \LaTeX{} se ocupa del formato del documento. El usuario no tiene que preocuparse de hacer saltos de página, justificaciones, sangrías, referencias, etc.
		\item[Versátil] se puede hacer casi cualquier cosa
		\item[Flexible] permite al usuario crear nuevos comandos y entornos 
		\item[Actualizado] \LaTeX{} es mejorado constantemente de forma altruista.
	\end{description}
\end{frame}


\begin{frame}[t,fragile]
    \frametitle{¿Cómo funciona?}
\begin{block}{}
	\begin{itemize}
        \setlength{\itemsep}{0.5\baselineskip}
        \item Escribimos un fichero de texto con el contenido y órdenes
        \item \LaTeX{} lo procesa y da como resultado un fichero (PDF) formateado
    \end{itemize}
\end{block}

\begin{exampleblock}{Ejemplo}
\begin{lstlisting}{language=tex}
\documentclass{article}
\begin{document}
    Consideremos una función \emph{continua} $f$. Su integral es...
\end{document}
\end{lstlisting}

    Consideremos la función \emph{continua} $f(x)=\cos(x)$. Su integral es...
\end{exampleblock}
\end{frame}



\begin{frame}[t]
\frametitle{Ventajas e inconvenientes}
\vspace{-2em}
\begin{columns}
\begin{column}[t]{0.5\textwidth}
	\begin{block}{Ventajas}
	\begin{itemize}
		\item Composición de fórmulas
		\item Calidad de imprenta
		\item Facilidad para gestionar bibliografías, 
		notas, referencias, etc.
		\item Muchos paquetes adicionales
		\item Independiente de la plataforma: Unix, Windows, OSX,...
		\item Software libre
		\item Salida PDF, Postscript,...
		\item Separación de contenido y forma
	\end{itemize}
	\end{block}
\end{column}
\begin{column}[t]{0.5\textwidth}
	\begin{block}{Inconvenientes}
	\begin{itemize}
		\item El diseño de un documento (nuevo) es difícil 
		si los predefinidos no se ajustan a lo que necesitamos
		\item Detección y manejo de errores
		\item Separación de contenido y forma
	\end{itemize}
	\end{block}
\end{column}
\end{columns}
\end{frame}


\begin{frame}[fragile]
\frametitle{Ayuda}
\begin{block}{}
	\begin{itemize}
	\addtolength{\itemsep}{0.5\baselineskip}
		\item Ayuda incluida en la instalación
		\item Listas de correo 
			\begin{itemize}
			\item Grupo de usuarios de \LaTeX{} de la UGR
			\url{https://groups.google.com/forum/#!forum/gul-ugr} 
			\item Lista de correo de Cervan\TeX{} 
			\url{http://www.rediris.es/list/info/es-tex.html} 
			\end{itemize} 
		\item Foros, blogs, grupos de noticias, etc.
			\begin{itemize}
			\item \url{https://es.sharelatex.com} (también editor online)
			\item \url{https://www.overleaf.com} (también editor online)
			\item \url{http://tex.stackexchange.com}
			\item \url{http://latex.org/forum/}
			\end{itemize}
	\end{itemize}
\end{block}
\end{frame}

\begin{frame}
\frametitle{¿Para que sirve?}
\begin{block}{Algunos usos}
\begin{itemize}
	\addtolength{\itemsep}{0.5\baselineskip}
		\item Artículos,
		\item exámenes, ejercicios,
		\item cartas, informes,
		\item libros, apuntes, 
		\item posters, presentaciones, etc.
	\end{itemize}
\end{block}
\end{frame}


\subsection{Paquetes}

\begin{frame}[t,fragile]
\frametitle{Paquetes}

\vspace{-1em}

\begin{columns}[t]
\begin{column}{0.5\textwidth}
\begin{block}{}
\begin{itemize}
\addtolength{\itemsep}{0.4\baselineskip}
\item \LaTeX{} es modular.
\item Hay módulos (\emph{paquetes}) que añaden posibilidades adicionales o modifican las definidas por defecto.
\item {\small \lstinline!\usepackage[opciones]{paquete}!} 
\end{itemize}
\end{block}
\end{column}

\begin{column}{0.5\textwidth}
\begin{block}{}
\begin{itemize}
\item Tipos (mathtpmx, etc.)
\item Aspecto (márgenes, cabeceras, etc.)

\item Gráficos (inclusión, construcción, posición, etc.)

\item Manejo de índices, glosarios, referencias, etc.
\item Currícula
\item Idioma
\item Hipervínculos, enlaces, etc.
\item Música
\item Cuadros, tablas, etc.

\end{itemize}
\end{block}
\end{column}
\end{columns}

\end{frame}




\section{Creación de un documento \LaTeX}


\begin{frame}
\frametitle{Ficheros \LaTeX}

\begin{description}[.bib, .bbl, .blg, .bst]
\item[.tex] El documento fuente es un fichero de texto que contiene tanto el texto como las instrucciones para formatear ese texto. Se puede crear con cualquier editor de textos.
\end{description}

\bigskip

Al compilar se obtienen varios documentos.

\medskip

\begin{description}[.bib, .bbl, .blg, .bst]
\addtolength\itemsep{0.5\baselineskip}
\item[.aux] Fichero auxiliar que contiene la información sobre las referencias, la bibliografía, el índice, etc.
\item[.dvi, .pdf] Posibles resultados de la compilación.

\item[.log] Mensajes del compilador.

\item[.toc, .lof, .lot] Información relativa a índices, lista de figuras y lista de tablas.

\item[.bib, .bbl, .blg, .bst] Ficheros relacionados con la bibliografía.

\end{description}

\end{frame}










\subsection{Estructura de un documento .tex}

\begin{frame}[fragile]
\frametitle{Partes de un documento .tex}

Cualquier  documento .tex tiene dos partes: el \emph{encabezamiento} y el \emph{cuerpo}

\begin{block}{Encabezamiento}
\begin{itemize}
\addtolength\itemsep{0.5\baselineskip}
	\item Contiene la información sobre los aspectos globales del
	documento: tipo de documento, tipo de letra, márgenes, 
	espacio entre líneas, etc. y los paquetes adicionales.
	


	\item Comienza con la declaración del tipo de documento:

	\lstinline!\documentclass[opciones]{tipo de documento}!
	

\end{itemize}
\end{block}

\begin{block}{Cuerpo}
\begin{itemize}
\addtolength\itemsep{0.5\baselineskip}
	\item Contiene el texto y los comandos
	para darle el formato deseado

	\item Se encuentra encerrado por los comandos \\
	\lstinline!\begin{document} ... \end{document}! 
\end{itemize}
\end{block}
\end{frame}

\subsection{Escritura en el documento fuente}
\begin{frame}[fragile]
\frametitle{Escritura en el documento fuente}
Hay que tener en cuenta que el aspecto final del documento \emph{no} se asemejará en absoluto al documento \texttt{.tex}

\begin{block}{}
En el documento fuente escribimos como si tuviésemos una línea infinita, que luego \LaTeX{} interpretará.
\begin{itemize}
\item \LaTeX{} finaliza las líneas donde considera más oportuno, justifica el texto por la derecha (realizando segmentación silábica) y realiza sangría por la izquierda al comienzo de cada párrafo
\item Para cambiar de párrafo debemos \emph{dejar una línea en blanco} o escribir \lstinline!\par!
\end{itemize}
\end{block}

\end{frame}

\begin{frame}[t,fragile]
\frametitle{Primer ejemplo}
\begin{exampleblock}{Nuestro primer texto en \LaTeX}
\begin{lstlisting}{language=tex}
\documentclass[a4paper]{article}
\usepackage[utf8]{inputenc}
\begin{document}
Pasos para instalar Latex en nuestro ordenador. 
Mejor dicho, Latex se escribe \LaTeX.
    
Los espacios en blanco no cuentan y si queremos 
empezar un párrafo nuevo sólo tenemos que dejar 
una línea en blanco. También podemos escribir 
fórmulas
\[
f(x)=\cos(x)+\frac{1}{x}
\]
\end{document}
\end{lstlisting}
\end{exampleblock}
\end{frame}


\begin{frame}[t,fragile]
\frametitle{Primer ejemplo - Cabecera}
\begin{lstlisting}
\documentclass[11pt]{article}
% Para escribir acentos...
\usepackage[utf8]{inputenc} 
% matemáticas, símbolos,... de la AMS
\usepackage{amsmath,amssymb,amsfonts,amsthm}
% selección del idioma
\usepackage[spanish]{babel}
% tamaño del papel
\usepackage[margin=3cm,a4paper]{geometry}
...
\begin{document}
...
\end{lstlisting}

\texttt{\%} se utiliza para añadir comentarios

\end{frame}

\begin{frame}[standout]
\frametitle{Compilación}

\begin{itemize}
\setlength{\itemsep}{\baselineskip}
\item ¿Cómo se compila?
\item ¿Errores?
\end{itemize}

\end{frame}



\subsection{Errores en la compilación}
\begin{frame}{Gestión de errores en la compilación}
Si \LaTeX{} encuentra errores en la compilación, para y se ``queja''. 

\medskip

\begin{block}{Posibles respuestas}
\begin{description}
\addtolength\itemsep{0.5\baselineskip}
\item[\keystroke{Enter}] le estamos diciendo olvida el error y haz lo que puedas. Puede ser necesario repetir el proceso varias veces
\item[\keystroke{x}+\keystroke{Enter}] \LaTeX{} para la compilación
\item[\keystroke{r}+\keystroke{Enter}] \LaTeX{} seguirá aunque encuentre errores
\item[\keystroke{e}+\keystroke{Enter}] \LaTeX{} para la compilación y nos manda al archivo fuente a la primera línea de código en la que encontró un error
\end{description}
\end{block}

Es fácil que la línea que señala \LaTeX{} no sea donde este se encuentre.

\end{frame}




\section{Primeros pasos}

\subsection{Comandos y entornos}

\begin{frame}[fragile]
\frametitle{Comandos}
\begin{alertblock}{Comandos}
\begin{itemize}
\setlength{\itemsep}{0.5\baselineskip}
\item Sirven para que \LaTeX{} realice una acción sencilla: cambiar de párrafo, escribir un símbolo, dejar un espacio\dots
\item Comienzan con \texttt{\textbackslash} y se escriben sólo con letras (distingue mayúsculas y minúsculas)
\item Pueden ser redefinidos y se pueden crear nuevos comandos
\item La sintaxis habitual es:

{\small \lstinline!\nombrecomando[opciones]{argumentos obligatorios}!}

\item \LaTeX{} ignora los espacios después de un comando
\end{itemize}
\end{alertblock}
\end{frame}

\begin{frame}[fragile]
\frametitle{Comandos}

\begin{exampleblock}{Ejemplos}
\begin{itemize}
\item \lstinline!\xi! escribe la letra griega xi: $\xi$
\item \lstinline!\hfill! inserta un espacio horizontal dinámico
\item \lstinline!\usepackage[spanish]{babel}! le dice a \LaTeX{} que cargue el paquete babel con la opción español
\end{itemize}
\end{exampleblock}


\begin{lstlisting}{language=tex}
Un documento contiene \textbf{texto} en negrita, 
letras griegas $\xi$. 

También podemos añadir una raya horizontal que se 
extienda hasta el final de la línea \hrulefill
\end{lstlisting}

\end{frame}



\begin{frame}
\frametitle{Entornos}
    \begin{alertblock}{Entornos}
        \begin{itemize}
            \item Son órdenes que sirven para que \LaTeX{} realice una acción compleja: crear una matriz, crear un página dentro de otra, escribir en varias columnas\dots
            \item Es necesario abrir el entorno y cerrarlo, la sintaxis es: \par \texttt{\textbackslash begin\{entorno\} \... \textbackslash end\{entorno\}}
            \item Los entornos también se pueden redefinir y se pueden crear otros nuevos
        \end{itemize}
    \end{alertblock}

    \begin{exampleblock}{Ejemplos}
        \begin{itemize}
            \item Entornos para escribir listas: itemize, enumerate
            \item Entornos para escribir tablas: table, array, matrix
            \item Entornos para situar el texto: center, flushleft, flushright
        \end{itemize}
    \end{exampleblock}

%Suele ser una buena estrategia cerrar los entornos justo después de abrirlos y luego continuar con el contenido del entorno.
\end{frame}

\subsection{Grupos}
\begin{frame}[fragile]
\frametitle{Grupos}


\begin{alertblock}{Grupo}
Es una parte bien delimitada del documento, con un inicio y un fin y que abarca todo lo que hay comprendido entre ambos
\medskip
\begin{itemize}
\item Para abrir un grupo utilizamos {\color{magenta}\{} y para cerrarlo {\color{magenta}\}}
\medskip
\item Los grupos se pueden anidar unos dentro de otros
\end{itemize}
\end{alertblock}
\end{frame}


\begin{frame}[fragile]
\frametitle{Grupos}
\begin{exampleblock}{Ejemplo}

\begin{lstlisting}{language=tex}
\textsc{Queremos escribir una frase en letras 
mayúsculas pequeñas {\color{blue} y una parte 
dentro de ella en \textbf{azul}} y a su vez 
otras partes  en \textbf{negrita} y otra más 
{\Large grande}}
\end{lstlisting}

\medskip

\includegraphics[width=\linewidth]{text.pdf}
\end{exampleblock}
\end{frame}





\subsection{Líneas, párrafos y páginas}


\begin{frame}[fragile]
\frametitle{Líneas y párrafos}
\begin{alertblock}{Espacios y párrafos}
\begin{itemize}
\addtolength\itemsep{0.5\baselineskip}
\item Uno o más espacios son tratados como un espacio.
\item También se trata como un espacio el salto de línea.

\item Varias líneas en blanco separan los párrafos.
\item El comando \lstinline!\par! tiene el mismo efecto.

\item \lstinline!\newline! inicia una nueva línea sin completar la línea en curso
\item \lstinline!\linebreak[opcion]! inicia una nueva línea justificando la línea en curso
\end{itemize}
\end{alertblock}

\end{frame}


\begin{frame}[t,fragile]
\frametitle{Alineación de párrafos}



\begin{block}{Alinear}
Se pueden alinear a izquierda o derecha párrafos usando
\begin{lstlisting}{language=tex}
\begin{flushleft}
Alineado a la izquierda\ldots
\end{flushleft}
\begin{flushright}
\ldots alineado a la derecha.
\end{flushright}
\end{lstlisting}
\end{block}

\begin{block}{Centrar párrafos}
Se pueden centrar párrafos con 
\begin{lstlisting}{language=tex}
\begin{center}
Esto es un texto centrado
\end{center}
\end{lstlisting}
\end{block}

\end{frame}


\begin{frame}[fragile]
\frametitle{Párrafos}

\begin{block}{}
\begin{itemize}
\addtolength\itemsep{0.5\baselineskip}
\item Hay entornos (quote, quotation, verse) para escribir algunos tipos de párrafos particulares

\item Se puede cambiar el espacio entre líneas de varias formas. Se recomienda usar el paquete \texttt{setspace} %(aunque también se pueden cambiar el valor de linespread o baselinestrecth)
\begin{lstlisting}
\usepackage{setspace}
\onehalfspacing % linea y media
\doublespacing % doble espacio
\end{lstlisting}

\item 
\LaTeX{} realiza una sangría a la izquierda al comienzo de cada nuevo párrafo. Si se quiere evitar se utiliza el comando \lstinline!\noindent!
\end{itemize}
\end{block}
\end{frame}



\begin{frame}[fragile]
\frametitle{Espacios, párrafos y páginas}

\begin{block}{Saltos de página}
\begin{itemize}
\addtolength\itemsep{0.5\baselineskip}
\item \lstinline!\newpage! inicia una nueva página sin completar la página en curso
\item \lstinline!\clearpage! produce un efecto similar al comando anterior ubicando los objetos ``flotantes'' (como tablas o gráficos) en una nueva página sin texto alguno
\end{itemize}
\end{block}


\end{frame}



\subsection{Símbolos especiales}
\begin{frame}{Símbolos especiales}
\begin{block}{Símbolos reservados}

Algunos caracteres tienen una utilidad especial para \LaTeX{} y su uso está reservado. Todos se pueden escribir anteponiendo una barra invertida salvo la propia barra invertida (\textbackslash{}\textbackslash indica línea nueva)
\begin{description}[\^{} y \_{}]
\addtolength\itemsep{0.4\baselineskip}
\item[\${}] Declarar el modo matemático {\color{rojoclaro}$\backslash$}{\color{blue}\$}
\item[\{{} \}{}] Iniciar y finalizar grupos {\color{magenta}{\color{rojoclaro}$\backslash$}\{\qquad{\color{rojoclaro}$\backslash$}\}}
\item[\#{}] Indicar el número de un argumento {\color{rojoclaro}$\backslash$}{\color{magenta}\#}
\item[\%{}] Hacer que \LaTeX{} ignore una línea de código {\color{rojoclaro}$\backslash$}{\color{magenta}\%}
\item[\&{}] Separar elementos de una tabla o una fórmula {\color{rojoclaro}$\backslash$}{\color{magenta}\&}
\item[\textbackslash{}] Inicio de cualquier comando {\color{blue}\$}{\color{rojoclaro}$\backslash$}backslash{\color{blue}\$}
\item[\^{} y \_{}] Escribir super y subíndices {\color{rojoclaro}$\backslash$\^ \qquad $\backslash$ \_}
\item[\~{}] ``Pegar'' palabras {\color{rojoclaro}$\backslash$\~\quad}
\end{description}
\end{block}
\end{frame}

\begin{frame}{Símbolos especiales}
\begin{block}{Símbolos ortográficos}
\begin{itemize}
\addtolength\itemsep{0.5\baselineskip}
\item Es mejor usar el paquete \emph{inputenc} con la codificación adecuada que escribir el comando necesario para cada símbolo.

\item ¿Cómo se escriben las <<comillas>>, ``comillas''?

\item ¿Y los puntos suspensivos...?

\item ¿Y los ordinales? 1\sptext{o} (¿o es 1º?) 
\end{itemize}
\end{block}

\end{frame}




\begin{frame}[fragile]
\frametitle{División de palabras}

\begin{block}{}
\begin{itemize}
\addtolength\itemsep{0.5\baselineskip}
\item \LaTeX{} se encarga de la división de palabras al final de línea cuando sea necesario

\item Se puede indicar como dividir una palabra concreta usando \textbackslash{}-

\item El comando \lstinline!\hyphenation{pa-la-bra1, pa-la-bra2,...}! en la cabecera vale para todo el documento

\item El paquete babel hace, entre otras cosas, que \LaTeX{} use los patrones de guionado del lenguaje seleccionado

\end{itemize}
\end{block}

\end{frame}








\subsection{Tipos y colores}

\begin{frame}[fragile]
\frametitle{Tipos}
\begin{block}{Familias de tipos de letra}
%Hay tres tipos de letra:
\begin{description}[\texttt{Máquina de escribir}]
\item[Texto normal] \lstinline!\textrm{Texto}! \quad $\rightsquigarrow$ \textrm{Texto}
\item[\textsf{Sanserif o sin adornos}] \lstinline!\textsf{Texto}! \quad $\rightsquigarrow$ \textsf{Texto}
\item[\texttt{Máquina de escribir}] \lstinline!\texttt{Texto}! \quad $\rightsquigarrow$ \texttt{Texto}
\end{description}
\end{block}


\bigskip

\begin{block}{Perfiles}
\begin{description}[\texttt{Máquina de escribir}]
\item[Recto]  \lstinline!\textup{Texto}! \quad $\rightsquigarrow$ \textup{Hola}
\item[Itálica] \lstinline!\textit{Texto}! \quad $\rightsquigarrow$ \textit{Hola}
\item[Inclinado]  \lstinline!\textsl{Texto}! \quad $\rightsquigarrow$ \textsl{Hola}
\item[Versalita]  \lstinline!\textsc{Texto}! \quad $\rightsquigarrow$ \textsc{Hola}
\end{description}
\end{block}

\end{frame}


\begin{frame}{Tipos}

\begin{block}{Grosor}
\begin{description}[\texttt{Máquina de}]
\item[Normal]  \texttt{\textbackslash{}textmd\{Texto\}} \quad  $\rightsquigarrow$ \textmd{hola}  
\item[Grueso]  \texttt{\textbackslash{}textbf\{Texto en negritas\}} \quad $\rightsquigarrow$ \textbf{hola}
\end{description}
\end{block}

\bigskip

\begin{block}{Otras formas de destacar texto}
\begin{description}[\texttt{Máquina de}]
\item[Resaltar] \texttt{\textbackslash emph\{Texto a resaltar\}} \quad $\rightsquigarrow$ \emph{hola}
\item[Subrayar] \texttt{\textbackslash{}underline\{Texto subrayado\}} \quad $\rightsquigarrow$ \underline{hola}
\end{description}
\end{block}

\end{frame}



\begin{frame}[fragile]
\frametitle{Tipos}

\begin{block}{Tamaño de letra}
\begin{lstlisting}{language=tex}
{\tiny Hay} {\footnotesize unos} {\small pocos} 
{\normalsize tamaños} {\large de} {\Large letra} 
{\huge en} {\Huge \LaTeX} {\huge que} {\Large se} 
{\large ponen} {\normalsize con} {\small los}
{\footnotesize comandos} {\tiny siguientes}
\end{lstlisting}
\medskip

{\tiny Hay} {\footnotesize unos} {\small pocos} 
{\normalsize tamaños} {\large de} {\Large letra} 
{\huge en} {\Huge \LaTeX} {\huge que} {\Large se} 
{\large ponen} {\normalsize con} {\small los}
{\footnotesize comandos:}

\texttt{
 {\tiny$\backslash$tiny}
{\scriptsize$\backslash$scriptsize}
{\footnotesize$\backslash$footnotesize}
{\small$\backslash$small}}

\medskip

\texttt{\normalsize$\backslash$normalsize}

\texttt{{\large$\backslash$large}
{\Large$\backslash$Large}
{\LARGE$\backslash$LARGE}
{\huge$\backslash$huge}
{\Huge $\backslash$Huge}}
\end{block}
\end{frame}


\begin{frame}{Colores}
\begin{block}{Colores}
\begin{itemize}
\addtolength{\itemsep}{0.5\baselineskip}
\item Es necesario cargar el paquete \texttt{color} o \texttt{xcolor}

\textbackslash{}usepackage[pdftex,usenames,dvipsnames]\{color\}

\item \textbackslash{}textcolor\{Red\}\{Texto\} $\ \rightsquigarrow$ \textcolor{Red}{Rojo}

\item \textbackslash{}textcolor[rgb]\{0.89,0.67,0.31\}\{Texto\} $\ \rightsquigarrow$ \textcolor[rgb]{0.89,0.67,0.31}{Otro color}

\end{itemize}
\end{block}
\end{frame}




\subsection{Listas}

\begin{frame}[fragile,t]
\frametitle{Listas}
\framesubtitle{Listas numeradas}
Existen tres entornos en \LaTeX{} para escribir listas: \alert<2->{enumerate}, itemize y description.

\bigskip \bigskip

\pause
\textbf{Entorno enumerate}


\begin{columns}
\begin{column}[]{0.5\textwidth}
\begin{lstlisting}{language=tex}
\begin{enumerate}
   \item Primer ítem,
   \item segundo ítem, y
   \item tercer ítem.
\end{enumerate}	
\end{lstlisting}
\end{column}
\begin{column}[]{0.5\textwidth}
\begin{enumerate}
   \item Primer ítem,
   \item segundo ítem, y
   \item tercer ítem.
\end{enumerate}	
\end{column}
\end{columns}


\end{frame}

\begin{frame}[fragile,t]
\frametitle{Listas}
\framesubtitle{Listas con viñetas}
Existen tres entornos en \LaTeX{} para escribir listas: enumerate, \alert<2->{itemize} y description.

\bigskip \bigskip

\pause

\textbf{Entorno itemize}



\begin{columns}
\begin{column}[]{0.5\textwidth}
\begin{lstlisting}{language=tex}
\begin{itemize}
   \item Primer ítem,
   \item segundo ítem, y
   \item tercer ítem.
\end{itemize}	
\end{lstlisting}
\end{column}
\begin{column}[]{0.5\textwidth}
\begin{itemize}
   \item Primer ítem,
   \item segundo ítem, y
   \item tercer ítem.
\end{itemize}	
\end{column}
\end{columns}


\end{frame}


\begin{frame}[fragile,t]
\frametitle{Listas}
\framesubtitle{Listas descriptivas}
Existen tres entornos en \LaTeX{} para escribir listas: enumerate, itemize y
\alert<2->{description}.

\bigskip \bigskip

\pause
\textbf{Entorno description}


\begin{columns}
\begin{column}[]{0.48\textwidth}
\begin{lstlisting}{language=tex}
\begin{description}
   \item[Curso] Dirección o carrera.
   \item[Alumno] Discípulo, respecto de su maestro...
   \item[Maestro] Dicho de un irracional.
\end{description}	
\end{lstlisting}
\end{column}
\begin{column}[]{0.48\textwidth}
\begin{description}
   \item[Curso] Dirección o carrera.
   \item[Alumno] Discípulo, respecto de su maestro...
   \item[Maestro] Dicho de un irracional.
\end{description}
\end{column}
\end{columns}


\end{frame}


\begin{frame}[fragile,t]
\frametitle{Listas}

Las listas se pueden anidar


\bigskip \bigskip

\begin{columns}
\begin{column}[]{0.5\textwidth}
\begin{lstlisting}{language=tex}
\begin{itemize}
  \item Varias cosas:
    \begin{enumerate}
       \item Una;
       \item otra;
       \item la última.
    \end{enumerate}
  \item segundo ítem y
  \item tercer ítem.
\end{itemize}
\end{lstlisting}
\end{column}
\begin{column}[]{0.5\textwidth}
\begin{itemize}
   \item Varias cosas:
   	\begin{enumerate}
	\item Una;
	\item otra;
	\item la última.
	\end{enumerate}
   \item segundo ítem y
   \item tercer ítem.
\end{itemize}
\end{column}
\end{columns}

\end{frame}


\begin{frame}[t]
\frametitle{Listas}

\begin{block}{¿Y después?}
\begin{enumerate}
\item El formato, la numeración, las vi\~{n}etas, el espaciado, sangrado, etc. se pueden modficar.
\item Todas las listas que hemos visto son un caso particular del entorno ``list''. Se pueden definir listas personalizadas.
\end{enumerate}
\end{block}

\bigskip \bigskip

\begin{block}{Ejercicios}
\begin{enumerate}
\item Prueba a anidar listas de diferentes tipos
\item ¿Qué ocurre si anidas m\'{a}s de cuatro?
\end{enumerate}
\end{block}
\end{frame}



\subsection{Matemáticas}

\begin{frame}[fragile]
\frametitle{Matemáticas}
Podemos escribir fórmula \alert{en línea}, $\sqrt{x+y}$ o en centradas en una línea separada
\[
\frac{\partial f}{\partial x} (x,y)= \sum_{n=1}^{\infty} \oint_{a}^{b} \frac{x}{1+x^2}\, \mathrm{d}x
\]
\begin{lstlisting}
Podemos escribir fórmula \alert{en línea}, 
$\sqrt{x+y}$ o en centradas en una línea separada
\[
f(x,y)= \sum_{n=1}^{\infty} \int_{a}^{b} 
\frac{x}{1+x^2}\, \mathrm{d}x
\]
\end{lstlisting}

\end{frame}






\subsection{Columnas}

\begin{frame}[t]
\frametitle{Columnas}

\begin{alertblock}{El paquete multicols}
{\small
\begin{multicols}{3}
\LaTeX{} trae incorporada la posibilidad de escribir a una o dos columnas. Sus posibilidades son limitadas.

\columnbreak

Es mucho mejor utilizar alguno de los paquetes dedicados a tal efecto. Hay muchos, pero uno de los m\'{a}s cómodos de usar es multicols.

\columnbreak

Las longitudes \texttt{columsep}, \texttt{columnseprule} y \texttt{multicolsep} permiten personalizar este entorno.
\end{multicols}
}

{\scriptsize
\begin{multicols}{4}
\blindtext
\end{multicols}
}
\end{alertblock}

\end{frame}



\begin{frame}[fragile,t,shrink]
\frametitle{Columnas}

{\scriptsize
\begin{tcolorbox}[colframe=blue!30!white,colback=white,arc=0mm,boxrule=1pt]
\vspace{-1em}
\begin{lstlisting}{language=tex}
\usepacakge{multicol}

\begin{multicols}{3}
% \columnsep = 3mm % separación entre columnas
% \columnseprule = 0.4pt % grosor de la línea de separación
% \multicolsep = 12pt plus 4pt minus 3pt  % separación del resto

\LaTeX{} trae incorporada la posibilidad de...

\columnbreak

Es mucho mejor utilizar alguno...
\end{multicols}

\begin{multicols}{4}
Lorem ipsum...
\end{multicols}
\end{lstlisting}
\end{tcolorbox}
}
\end{frame}



\subsection{¿Y después?}
\begin{frame}
\frametitle{¿Y después?}




Índices, índices de términos, referencias cruzadas, pies de página, bibliografías, cuadros, teoremas, inclusión de código, música, tipos de documentos, presentaciones, gráficos,...
\end{frame}


\begin{frame}[standout]

{\Large ¿Preguntas?}

\bigskip

{\large Gracias}

\end{frame}

\end{document}
