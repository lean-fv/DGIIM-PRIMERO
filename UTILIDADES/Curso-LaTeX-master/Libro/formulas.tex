\chapter{Fórmulas}

Básicamente hay dos tipos de fórmulas.
\begin{itemize}
\item Aquellas que van insertadas en el texto, como por ejemplo $2^{x+y}\int_a^b e^{\frac{x^2}{2}}\lim_{x\to 1}x^{x-1}$.
\item Otras que se ponen en modo pantalla (\emph{display}): \[\max\{2^{x+y}\int_a^b e^{\frac{x^2}{2}}\lim_{x\to 1}x^{x-1},1\}.\]
Compárese esta última con 
\[\max\left\{2^{x+y}\int_a^b e^{\frac{x^2}{2}}\lim_{x\to 1}x^{x-1},1\right\}.\]
\end{itemize}

También podemos poner fórmulas con etiquetas,
\begin{equation}\label{formula} %acabamos de ponerle una etiqueta a la fórmula 
\sum_{i=1}^n i=\frac{n(n+1)}2,
\end{equation}
para poder referirnos a ellas más tarde (por ejemplo: la fórmula \ref{formula} se verifica para todo $n$ entero positivo). %con ref hacemos referencia a una etiqueta definida en otro sitio del texto
