\documentclass[11pt]{article}

\usepackage[T1]{fontenc}
\usepackage[utf8]{inputenc}
\usepackage[a4paper,margin=3cm]{geometry}


% paquetes relacionados con matemáticas
\usepackage{amsmath,amssymb,amsthm}
\usepackage{mathtools}

\numberwithin{equation}{section}
\allowdisplaybreaks[2]

\DeclareMathOperator{\deter}{determinante}
\DeclareMathOperator{\dist}{dist}

% idioma
\usepackage[spanish,es-lcroman]{babel}


% teoremas
\theoremstyle{plain}
\newtheorem{teorema}{Teorema}[section]
\newtheorem{proposicion}[teorema]{Proposición}
\newtheorem{lema}[teorema]{Lema}
\newtheorem{corolario}[teorema]{Corolario}
\theoremstyle{definition}
\newtheorem{definicion}[teorema]{Definición}
\newtheorem{ejemplo}[teorema]{Ejemplo}
\theoremstyle{remark}
\newtheorem{remark}[teorema]{Observación}

% Comandos propios

\newcommand{\abs}[1]{\ensuremath\left\vert #1 \right\vert}
\newcommand{\norm}[1]{\ensuremath\left\Vert #1 \right\Vert}


\author{El Autor}
\title{El título}



\begin{document}

\maketitle

\section{Definición}



Sea $I$ un intervalo y sean $a$, $b \in I$ con $a<b$. Si $x\in [a,b]$, $x$ se puede escribir como \emph{combinación convexa} de $a$ y $b$:
\[
 x= \frac{b-x}{b-a}\cdot a + \frac{x-a}{b-a}\, b.
\]
Otras fórmulas
\[
 \lim_{t \to 0} \frac{f(a+t,b)-f(a,b)}{t}= \frac{\partial f}{\partial x}(a,b).
\]



\begin{definicion}\label{def:convexa}
 Sea $I$ un intervalo y sea $f \colon I \rightarrow \mathbb{R}$.
 \begin{enumerate}
  \item La función $f$ es \emph{convexa} si para cualesquiera $a$, $b \in I$ con $a<b$ y cualquier $t \in [0,1]$ se cumple que
        \[
         f\left((1-t)a+tb\right) \leq (1-t)f(a)+tf(b).
        \]
  \item La función $f$ es \emph{cóncava} (resp. estrictamente cóncava) si $-f$ es convexa (resp. estrictamente cóncava).
 \end{enumerate}
\end{definicion}

\begin{ejemplo}
 La función $f(x)=x^2$ es convexa en todo $\mathbb{R}$. En efecto, si $a$, $b \in \mathbb{R}$ y $t \in [0,1]$,
 \begin{align*}
  f\left( (1-t)a+tb \right) & = (1-t)^2 a^{2}+t^{2}b^{2} +2t(1-t)ab, \\
  (1-t)f(a)+tf(b)           & = (1-t)a^2+tb^2.
 \end{align*}
 Para ver cuál es mayor, calculamos la diferencia:
 \begin{align*}
  (1-t)f(a) & +tf(b)  - \bigl( f\left( (1-t)a+tb \right) \bigr) \\
  \begin{split}
   & = (1-t)a^2+tb^2 - \left( (1-t)^2 a^{2}+t^{2}b^{2} +2t(1-t)ab \right) \\
   & = t(1-t) \left( a-b \right)^2 \geq 0.
  \end{split}
 \end{align*}
\end{ejemplo}

\subsection{El conjunto de las funciones convexas}

Veamos algunas propiedades del conjunto de las funciones convexas (ver definición~\ref{def:convexa}).

\begin{proposicion}
 Sean $f$, $g \colon I \rightarrow \mathbb{R}$ dos funciones convexas y
 sea $\alpha>0$. Entonces
 \begin{enumerate}
  \item $f+g$ es convexa,
  \item $\alpha f$ es convexa, y
  \item $\max \{f ,g\}$ es convexa.
 \end{enumerate}
\end{proposicion}
\begin{proof}
 \begin{enumerate}
  \item Sean $a$, $b\in I$, y sea $t\in [0,1]$, entonces
        \begin{equation}
         \begin{split}
          (f+g) \left( (1-t)a+tb \right) & = f \left( (1-t)a+tb \right) + g \left( (1-t)a+tb \right)\\
          & \leq (1-t)f(a)+tf(b)+(1-t)g(a)+t g(b) \\
          & = (1-t)(f+g)(a) + t(f+g)(b).
         \end{split}
        \end{equation}
  \item Sean $a$, $b\in I$, y sea $t\in [0,1]$, entonces
        \begin{equation}
         \begin{aligned}
          (\alpha f) \left( (1-t)a+tb \right) & = \alpha f \left( (1-t)a+tb \right)     \\
                                              & \leq \alpha (1-t)f(a)+ \alpha t f(b)    \\
                                              & = (1-t)(\alpha f)(a) + t (\alpha f)(b).
         \end{aligned}
        \end{equation}
  \item Sean $a$, $b\in I$, y sea $t\in [0,1]$, entonces
        \[
         \begin{aligned}
          \max \{ f,g\} \left( (1-t)a+tb \right) & = \max \left\{ f\left( (1-t)a+tb) \right),g \left((1-t)a+tb\right) \right\} \\
                                                 & \leq \max \left\{ (1-t)f(a)+tf(b),(1-t)g(a)+t g(b) \right\}                 \\
                                                 & = (1-t) \max\{f,g\}(a) + t \max\{ f,g\} (b). \qedhere
         \end{aligned}
        \]
 \end{enumerate}
\end{proof}


\section{Segunda sección}


\begin{teorema}[O. Stolz] \label{th:cc}
 Sea $I$ un intervalo abierto y sea $f \colon I \rightarrow$ una función convexa.
 \begin{enumerate}
  \item La función $f$ tiene derivadas laterales en todos los puntos. En particular, $f$ es continua.
  \item $f'_{+}$ y $f'_{-}$ son funciones crecientes.
 \end{enumerate}
\end{teorema}

\section{Tercera sección}

\begin{teorema}[de la función inversa]\label{th:inversa}
 Sean $A\subset \mathbb{R}^N$ un
 abierto y $f\colon A\rightarrow \mathbb{R}^{N}$ un campo vectorial de clase
 $\mathcal{C}^1$. Supongamos que existe $a\in A$ tal que
 \[
  \deter \bigl (J_f(a)\bigr )=\det
  \left(
  \begin{matrix}
    D_1 f_1 (a) & \dotsc & D_{N} f_{1}(a) \\
    \hdotsfor[2]{3}                       \\
    D_1f_N(a)   & \dotsc & D_N f_N(a)
   \end{matrix}
  \right)
  \neq 0.
 \]
 Entonces existe un entorno abierto $U$ de $a$ contenido en $A$ tal
 que $V:=f(U)$ es un abierto de $\mathbb{R}^N$, $f$ es un difeomorfismo de
 clase $\mathcal{C}^1$ de $U$ sobre $V$, y para cada $x\in U$ se
 verifica
 \[
  J_{f^{-1}}\bigl (f(x)\bigr )=
  \begin{pmatrix}
   D_1f_1(x) & \dotsc & D_Nf_1(x) \\
   \hdotsfor[2]{3}                \\
   D_1f_N(x) & \dotsc & D_Nf_N(x)
  \end{pmatrix}^{-1}.
 \]
\end{teorema}

\begin{teorema}[de la convergencia monótona]
 Sea $(\Omega,\mathcal{A},\mu)$ un espacio de medida. Si $\{f_n\}$ es
 una sucesión monótona de funciones integrables en $\Omega$
 verificando que la sucesión de sus integrales $\big
  \{\int_{\Omega}f_n\  d\mu \big \}$ está acotada, entonces
 $\{f_n\}$ converge $\text{c.p.d.}$ a una función $f$ integrable en
 $\Omega$ y
 \[
  \int_{\Omega}f\, \mathrm{d}\mu=\lim \int_{\Omega}f_n\, \mathrm{d}\mu .
 \]
\end{teorema}

\end{document}
