% !TEX encoding = UTF-8 Unicode

\documentclass{article}

% Codificación (acentos)
\usepackage[utf8]{inputenc}

% Cargamos un paquete para cambiar la tipografía.
% Ver más opciones en https://tug.org/FontCatalogue/
\usepackage[sc]{mathpazo}
\linespread{1.05}         % Palladio needs more leading (space between lines)
\usepackage[T1]{fontenc}



%%% Matemáticas Paquetes AMS
\usepackage{amsfonts,amsmath,amssymb,amsthm} % Símbolos, teoremas de AMS
\usepackage{mathtools} % % Algunos añadidos y correcciones a amsmath
\usepackage{esvect} % Vectores con flechas

\numberwithin{equation}{section}
\allowdisplaybreaks[1] % 1-4, permite partir ecuaciones entre páginas

\DeclareMathOperator{\arcsen}{arc\,sen} 
\DeclareMathOperator{\dist}{distancia}
\DeclareMathOperator*{\limite}{límite}

\DeclarePairedDelimiter\abs{\lvert}{\rvert}
\DeclarePairedDelimiter\norma{\lVert}{\rVert}

% teoremas
\theoremstyle{plain}
\newtheorem{teorema}{Teorema}[section]
\newtheorem{prop}[teorema]{Proposición}
\newtheorem{coro}[teorema]{Corolario}
\newtheorem{lema}[teorema]{Lema}
\theoremstyle{definition}
\newtheorem{definicion}[teorema]{Definición}
\theoremstyle{remark}
\newtheorem*{observacion}{Observación} 




%%% Configuración del idioma (el valor por defecto es el último)
\usepackage[english,spanish]{babel}
\decimalpoint

% Paquete tikz para construir gráficos
\usepackage{tikz}
\usetikzlibrary{babel}

% Ajuste tipográficos
\usepackage[babel]{microtype}

% Números y unidades
\usepackage{siunitx}
\sisetup{%
  exponent-product={\ensuremath{\cdot}},
  inter-unit-product={\,},
  output-decimal-marker={.}
  }


% Tablas
\usepackage{booktabs} % ayuda para mejorar el aspecto de las tablas
\renewcommand{\arraystretch}{1.2} % aumenta la separación entre las filas de una tabla

% H para forzar figuras en un lugar
\usepackage{float}

% Paquetes para incluir código
%\usepackage{minted}
% \usepackage{listing}
  
 % Tamaño del papel
\usepackage[twoside,a4paper,margin=2.5cm]{geometry}


% Enlaces 
\usepackage{hyperref}
\usepackage{url}
\hypersetup{pdftitle={Introducción al lenguaje LaTeX para edición de textos académicos},
  pdfsubject={Curso de LaTeX}
  pdfauthor={Orientamat}
  pdfkeywords={LaTeX, Orientamat}
}


%%% Metadatos de este documento

\title{Matemáticas} % título
\author{Orientamat} % autor
\date{\today} % fecha

\begin{document}

\maketitle

\tableofcontents

\bigskip

\section{Introducción}

Una de las ventajas de \LaTeX{} es la facilidad para incluir fórmulas o ecuaciones, que pueden extenderse a lo largo de varias líneas, conteniendo símbolos, operadores y delimitadores.

Existen tres \emph{modos} en \LaTeX:
\begin{description}
    \item[Párrafo] Escribimos el texto como una sucesión de palabras separadas por líneas en blanco. El texto aparece justificado.
    \item[Sin justificar] Se escribe igual, pero \LaTeX{} lo escribe en una línea de izquierda a derecha. Es lo que pasa, por ejemplo, al usar \mbox{\textcolor{red}{el texto que sea, aunque no entre en la línea y quede una línea muy largar sin partir}}.
    \item[Modo matemático] Se pueden usar símbolos (raíces, fracciones,...), todas las letras se consideran símbolos, que aparecen en itálica, y los espacios no se tienen en cuenta. El espaciado depende del tipo de símbolo.
\end{description}

\section{¿Cómo escribir matemáticas?}

\subsection{Entornos matemáticos}

% $,\[ equation equation* label ref eqref 

Hay dos formas de incluir expresiones matemáticas en el texto:
\begin{itemize}
    \item En el mismo párrafo con el resto del texto o
    \item centrado en una línea, o varias, aparte.
\end{itemize}



Consideremos la ecuación 
\begin{equation} \label{eq:grado2}
    ax^2+bx+c=0,
\end{equation}
donde \(a\), \(b\) y \(c\) son números reales con $a \neq 0$. Su solución 
general es
\begin{equation} \label{eq:solucion}
    x=\frac{-b \pm \sqrt{b^2-4ac}}{2a}. \tag{Solución}
\end{equation}
El número de soluciones reales depende del valor del \emph{discriminante}, 
\[
    \Delta = b^2-4ac.
\]
Si el discriminante es negativo, entonces la ecuación~(\ref{eq:grado2}) no 
tiene soluciones reales.






\subsection{Espaciado horizontal}

Los espacios dentro del modo matemático \emph{no} se tienen en cuenta. El espacio dentro de las fórmulas es distinto al espacio en modo texto.


Sea $x=1,2$ o $3$ con sea $x=1$, $2$ o $3$ y sean $a$, $b\in \mathbb{R}$.


% \, \! \: \quad \qquad \hspace{1cm} ~


\subsection{Letras diversas}

%\mathrm \mathit \mathsf \mathbb \mathcal \mathfrak

% \displaystyle textstyle script scriptscript


\[ 
	f(z)=\int_{C(a,r)} \frac{f(w)}{w-z} \,\mathrm{d} w, \quad \forall\, z \in D(a,r).
\]


\section{Construcciones básicas}

\subsection{Subíndices y superíndices}

% ^ _ \prescript{}{} \substack



\subsection{Fracciones y binomios} 

% frac dfrac tfrac binom dbinom tbinom splitfrac cfrac


\[
\frac{1+x}{2-\frac{x}{x+1}} + \frac{1+\dfrac{1}{2}}{1-\frac{1}{2}}
\]
\[
  a_0+\cfrac{1}{a_1+\cfrac{1}{a_2+\cfrac{1}{a_3+\cdots}}}
\]    
\[ a=\frac{
          \splitfrac{xy + xy + xy + xy + xy}
                    {+ xy + xy + xy + xy}
}
{z} =\frac{
          \splitdfrac{xy + xy + xy + xy + xy}
                    {+ xy + xy + xy + xy}
    }{z} 
\]



\subsection{Raíces}

% \sqrt \sqrt[3]{x} $\sqrt[\leftroot{2} \uproot{2} 6+x]{3}$	


\subsection{Puntos suspensivos}

% dots ldots cdots dotsc 


\subsection{Acentos y gorros}

%\acute hat bar tilde widehat widetilde mathring

% esvect vv en vez de \vec

% overline underline

% overbrace underbrace
\[
\overbrace{a+b+\underbrace{c+e+d}+f+g}^{n}+\underbrace{a+b}_{(2)}
\]

\subsection{Integrales, sumatorios, productos}

% int sum prod 

\subsection{Conjuntos}

% subset susbseteq in notin

\subsection{Texto en fórmulas}

% \text intertext


\subsection{Letras griegas}    

%alpha beta gamma delta epsilon varepsilon pi nu omega tau lambda Phi Psi


    
\subsection{Operadores}

% sin cos min exp ker gcd hom arctan lim log tan arctan

% sen max lim tg arctg

% DeclareMathOperator DeclareMathOperator*

%DeclareMathOperator*{\limite}{límite}



\subsection{Símbolos}

% binarios, relaciones 
% : colon | mid

\subsubsection{Operaciones binarias}

%+ -  *

\subsubsection{Relaciones}

% < leq le neq  > approx


\subsubsection{Flechas}

%leftarrow rightarrow leftrightarrow Rightarrow implies iff


\subsubsection{Símbolos derivados de letras}

% in partial ell exists re im imath jmath

\subsubsection{Símbolos de tamaño variable}

% bigcap bigcup sum prod int


\subsection{Delimitadores}

% ( [ { | || langle 

\subsubsection*{Una aplicación: valor absoluto y norma}

%\newcommand*\abs[1]{\left\lvert#1\right\rvert}
%\newcommand*\norma[1]{\left\lVert#1\right\rVert}

%\DeclarePairedDelimiter\abs{\lvert}{\rvert}
%\DeclarePairedDelimiter\norma{\lVert}{\rVert}





\subsection{Flechas extensibles}

%\xleftarrow[abajo]{arriba} \xrightarrow[abajo]{arriba}


\subsection{Matrices}

% matrix pmatrix bmatrix vmatrix smallmatrix

% array{lcr}


\subsection{Unidades}

% siunitx \num{123456.0123456} \SI{1000}{\km\per\hour}


\section{Ecuaciones en varias líneas}

% equation split


%\begin{itemize}
%	\item Si le añadimos una $*$ al entorno, obtenemos la versión sin numerar;
%	\item \verb+\\+ se utiliza para partir la línea;
%	\item \verb+\\+ se usa en todas las líneas, menos en la última;
%	\item \verb+&+ se usa, en los entornos que lo permitan, para indicar un punto común donde se alinean las expresiones;
%	\item No se pueden dejar líneas en blanco;
%	\item No se dejan líneas en blanco antes de una ecuación.
%\end{itemize}

\subsection{Una ecuación en varias líneas}


\subsubsection*{Una expresión alineada: el entorno split}


\subsubsection*{Una ecuación sin alinear: el entorno multline}

\subsection{Varias ecuaciones}

\subsubsection*{Sin alinear: el entorno gather}


\subsubsection*{Varias ecuaciones con alineamiento}

\subsubsection*{flalign}


\subsection{Entornos subsidiarios}
\[
\left.
\begin{aligned}[c]
 x & = 3 + \mathbf{p} + \alpha\\
      y & = 4 + \mathbf{q}\\
      z & = 5 + \mathbf{r}\\
      u & = 6 + \mathbf{s} \\
      \beta & = 1
\end{aligned} \right\}
\text{\quad usando\quad}
\left[
\begin{gathered}
      \mathbf{p} = 5 + a + \alpha\\
      \mathbf{q} = 12\\
      \mathbf{r} = 13\\
      \mathbf{s} = 11 + d
\end{gathered}
\right.
\]


\[	
f(x) = \begin{cases}
			x^2, & \text{si $x>0$,} \\
			1-x, & \text{si $x \leq 0$.}
	   \end{cases}
\]			


\subsection{Miscelánea} \label{sec:misc-ecuaciones}

% displaybreak allowdisplaybreaks intertext shortintertext


%	\numberwithin{equation}{section}	

% tag notag




\section{Teoremas y demostraciones}

%\newtheorem{entorno}{Nombre}[nivel para la numeración]
%\newtheorem{entorno}[misma numeración que]{Nombre}


%\theoremstyle{plain}
%\newtheorem{teorema}{Teorema}[section]
%\newtheorem{coro}[teorema]{Corolario}
%\newtheorem{lema}[teorema]{Lema}
%\theoremstyle{definition}
%\newtheorem{definicion}[teorema]{Definición}
%\theoremstyle{remark}
%\newtheorem*{observacion}{Observación} 


\begin{definicion}
Sea $f \colon [a,b] \to \mathbb{R}$ una función. Diremos que $f$ es \emph{lipschitziana} si existe una constante $K$ tal que 
\[
\left| f(x)-f(y) \right| \leq M \left| x-y \right|, \quad \forall x,y \in [a,b].
\]
\end{definicion}

\begin{prop}
Las funciones lipschitzianas son continuas.
\end{prop}
\begin{proof}
Aquí va la demostración completa, con sus cuentas, algún $\varepsilon$ y algún $\delta$.
\end{proof}

\begin{teorema}[de los ceros de Bolzano]
Sea $f \colon [a,b] \to \mathbb{R}$ una función cotinua verificando que $f(a)f(b)<0$. Entonces existe $c \in (a,b)$ tal que $f(c)=0$.
\end{teorema}


\section{Entornos flotantes}



\begin{thebibliography}{99}
\raggedright

\bibitem{AMUG} American Mathematical Society and the \LaTeX3 Project:
  \emph{User's Guide for the \textnormal{\ttfamily amsmath} package},
  Version~2.$+$,
  \url{https://mirror.ctan.org/macros/latex/required/amsmath/amsldoc.tex} and
  \url{https://mirror.ctan.org/macros/latex/required/amsmath/amsldoc.pdf},
  2021.

\bibitem{BEZ} Javier Bezos, \emph{Ortotipografía y notaciones matemáticas}, \url{https://www.texnia.com/archive/ortomatem.pdf}.

\bibitem{CLSL} Scott Pakin:
  \emph{The Comprehensive \LaTeX{} Symbol List},
  \url{https://ctan.org/tex-archive/info/symbols/comprehensive/}.

\bibitem{LFC} The \LaTeX{} Font Catalogue, 
  \url{https://tug.org/FontCatalogue/allfonts.html} .

\bibitem{MML} George Gr\"atzer: \textit{More Math into \LaTeX},
   5th edition, Springer, New York, 2016.

\bibitem{SMG} American Mathematical Society: \emph{Short Math Guide v.2.0}, \url{http://mirrors.ctan.org/info/short-math-guide/short-math-guide.pdf}.

\end{thebibliography}


\end{document}
