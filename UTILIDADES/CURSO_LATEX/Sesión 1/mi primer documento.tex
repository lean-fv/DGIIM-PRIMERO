% Comentarios
%Se llevan a cabo mediante el simbolo %.

%Estructura basica de un comando:
%\comando[opcion_1,...opcion_n]{argumento_1}..{argumento_n}

% 1. ESTRUCTURA DEL DOCUMENTO ------------------------------------------
% Preámbulo

% Tipo de documento ----------------------------------------------------
\documentclass[a4paper,12pt]{article} %book, report, letter, beamer

% Argumentos opcionales: tamaño del papel, tamaño base de la tipografía.

% Concepto de paquete --------------------------------------------------
% Codificación
\usepackage[utf8]{inputenc} %Opcion inputenc mantiene configuracion entre dispositivos. utf8 es la estandar.

% Idioma
\usepackage[spanish]{babel}

% Tipografia
\usepackage[cmintegrals,cmbraces]{newtxmath}
\usepackage{ebgaramond-maths}
\usepackage[T1]{fontenc}

% Otros

% Referenciar. Ver mas opciones de \hiperref{}
\usepackage[backref]{hyperref}

% Título y autor de un documento. Fecha. Tabla de contenido
\title{Mi Primer Documento}
\author{Leandro Jorge Fernández Vega}
\date{4 de marzo, 2023}

% Cuerpo del documento -------------------------------------------------
%Entorno: comando especial que cuenta con parte de incio y de fin.

\begin{document}

% Necesario para cargar datos en el preambulo del titulo.
\maketitle
% Crea indice
\tableofcontents

% Generalidades --------------------------------------------------------
% Comandos, símbolos reservados (solos no se imprimen): $ $, {}, #, %, &, \, ^, _, ~
% Con \simbolo se imprimen 
% Compilación, archivos auxiliares. Errores

% 2. PARTES DE UN DOCUMENTO --------------------------------------------
% Partes, capítulos, secciones, párrafos, líneas y saltos de línea, guiones

%\part{Primera Parte} o \chapter{Capítulo} daría error pues no estan definidas esta estructura en el tipo de escrito que estamos haciendo.

\section{Inclusion de texto en un documento}

\subsection{Párrafos}

%Los saltos de linea se hacen dejando una linea en blanco.

Capítulo Primero 

Extracto del primer capitulo del Quijote \cite{Cer1605}
Que trata de la condición y ejercicio del famoso hidalgo D. Quijote de la Mancha

En un lugar de la Mancha, de cuyo nombre no quiero acordarme, no ha mucho tiempo que vivía un hidalgo de los de lanza en astillero, adarga antigua, rocín flaco y galgo corredor. Una olla de algo más vaca que carnero, salpicón las más noches, duelos y quebrantos los sábados, lentejas los viernes, algún palomino de añadidura los domingos, consumían las tres partes de su hacienda. El resto della concluían sayo de velarte, calzas de velludo para las fiestas con sus pantuflos de lo mismo, los días de entre semana se honraba con su vellori de lo más fino. Tenía en su casa una ama que pasaba de los cuarenta, y una sobrina que no llegaba a los veinte, y un mozo de campo y plaza, que así ensillaba el rocín como tomaba la podadera. Frisaba la edad de nuestro hidalgo con los cincuenta años, era de complexión recia, seco de carnes, enjuto de rostro; gran madrugador y amigo de la caza. Quieren decir que tenía el sobrenombre de Quijada o Quesada (que en esto hay alguna diferencia en los autores que deste caso escriben), aunque por conjeturas verosímiles se deja entender que se llama Quijana; pero esto importa poco a nuestro cuento; basta que en la narración dél no se salga un punto de la verdad.

Es, pues, de saber, que este sobredicho hidalgo, los ratos que estaba ocioso (que eran los más del año) se daba a leer libros de caballerías con tanta afición y gusto, que olvidó casi de todo punto el ejercicio de la caza, y aun la administración de su hacienda; y llegó a tanto su curiosidad y desatino en esto, que vendió muchas hanegas de tierra de sembradura, para comprar libros de caballerías en que leer; y así llevó a su casa todos cuantos pudo haber dellos; y de todos ningunos le parecían tan bien como los que compuso el famoso Feliciano de Silva: porque la claridad de su prosa, y aquellas intrincadas razones suyas, le parecían de perlas; y más cuando llegaba a leer aquellos requiebros y cartas de desafío, donde en muchas partes hallaba escrito: la razón de la sinrazón que a mi razón se hace, de tal manera mi razón enflaquece, que con razón me quejo de la vuestra fermosura, y también cuando leía: los altos cielos que de vuestra divinidad divinamente con las estrellas se fortifican, y os hacen merecedora del merecimiento que merece la vuestra grandeza. Con estas y semejantes razones perdía el pobre caballero el juicio, y desvelábase por entenderlas, y desentrañarles el sentido, que no se lo sacara, ni las entendiera el mismo Aristóteles, si resucitara para sólo ello. No estaba muy bien con las heridas que don Belianis daba y recibía, porque se imaginaba que por grandes maestros que le hubiesen curado, no dejaría de tener el rostro y todo el cuerpo lleno de cicatrices y señales; pero con todo alababa en su autor aquel acabar su libro con la promesa de aquella inacabable aventura, y muchas veces le vino deseo de tomar la pluma, y darle fin al pie de la letra como allí se promete; y sin duda alguna lo hiciera, y aun saliera con ello, si otros mayores y continuos pensamientos no se lo estorbaran.


\subsubsection{Líneas y saltos de línea}

%Caracteres \\ fuerzan un salto de linea pero no tabulan.
A un olmo seco

Al olmo viejo, hendido por el rayo \\
y en su mitad podrido, \\
con las lluvias de abril y el sol de mayo \\
algunas hojas verdes le han salido. \\

Antonio Machado.


\section{Segunda sección}

% 3. ENTORNOS: alineación, listas --------------------------------------

% Centrar
\begin{center}

A un olmo seco

Al olmo viejo, hendido por el rayo \\
y en su mitad podrido, \\
con las lluvias de abril y el sol de mayo \\
algunas hojas verdes le han salido. \\

Antonio Machado.

\end{center}

% Alinear a la izquierda
\begin{flushleft}

A un olmo seco

Al olmo viejo, hendido por el rayo \\
y en su mitad podrido, \\
con las lluvias de abril y el sol de mayo \\
algunas hojas verdes le han salido. \\

Antonio Machado.

\end{flushleft}

% Alinear a la derecha
\begin{flushright}

A un olmo seco

Al olmo viejo, hendido por el rayo \\
y en su mitad podrido, \\
con las lluvias de abril y el sol de mayo \\
algunas hojas verdes le han salido. \\

Antonio Machado.

\end{flushright}

% Cita
\begin{quote}
    la razón de la sinrazón que a mi razón se hace, de tal manera mi razón enflaquece, que con razón me quejo de la vuestra fermosura, y también cuando leía: los altos cielos que de vuestra divinidad divinamente con las estrellas se fortifican, y os hacen merecedora del merecimiento que merece la vuestra grandeza.
\end{quote}

% Listas

% Por puntos
\section{Obras de Cervantes}
\begin{itemize}
    
\item La Galatea (1585)
\item El ingenioso hidalgo don Quijote de la Mancha (1605)
\item Novelas ejemplares (1613)
\item El ingenioso caballero don Quijote de la Mancha (1615)
\item Los trabajos de Persiles y Sigismunda (1617)

\end{itemize}

% Enumerada
\begin{enumerate}
    \item La Galatea (1585)
    \item El ingenioso hidalgo don Quijote de la Mancha (1605)
    \item Novelas ejemplares (1613)
    \item El ingenioso caballero don Quijote de la Mancha (1615)
    \item Los trabajos de Persiles y Sigismunda (1617)
\end{enumerate}

% Decriptiva

\begin{description}

\item[1585] La Galatea
\item[1605] El ingenioso hidalgo don Quijote de la Mancha

\end{description}

% Anidada

    \begin{enumerate}
    \item La Galatea (1585)
        \begin{itemize}
            \item Primer elemento
            \item Segundo elemento
        \end{itemize}
    \item El ingenioso hidalgo don Quijote de la Mancha (1605)
    \item Novelas ejemplares (1613)
    \item El ingenioso caballero don Quijote de la Mancha (1615)
    \item Los trabajos de Persiles y Sigismunda (1617)
\end{enumerate}





% 4. TIPOGRAFÍA: estilos y tamaños -------------------------------------
\section{Tipografía}

Capítulo Primero 

\textbf{Que trata de la condición y ejercicio del famoso hidalgo D. Quijote de la Mancha}

\textsc{En un lugar de la Mancha}, de cuyo nombre no quiero acordarme, no ha mucho tiempo que vivía un hidalgo de los de lanza en astillero, adarga antigua, rocín flaco y galgo corredor. Una olla de algo más vaca que carnero, salpicón las más noches, duelos y quebrantos los sábados, lentejas los viernes, algún palomino de añadidura los domingos, consumían las tres partes de su hacienda. El resto della concluían sayo de velarte, calzas de velludo para las fiestas con sus pantuflos de lo mismo, los días de entre semana se honraba con su vellori de lo más fino. Tenía en su casa una ama que pasaba de los cuarenta, y una sobrina que no llegaba a los veinte, y un mozo de campo y plaza, que así ensillaba el rocín como tomaba la podadera. Frisaba la edad de nuestro hidalgo con los cincuenta años, era de complexión recia, seco de carnes, enjuto de rostro; gran madrugador y amigo de la caza. Quieren decir que tenía el sobrenombre de Quijada o Quesada (que en esto hay alguna diferencia en los autores que deste caso escriben), aunque por conjeturas verosímiles se deja entender que se llama Quijana; pero esto importa poco a nuestro cuento; basta que en la narración dél no se salga un punto de la verdad.

Es, pues, de saber, que este sobredicho hidalgo, los ratos que estaba ocioso (que eran los más del año) se daba a leer libros de caballerías con tanta afición y gusto, que olvidó casi de todo punto el ejercicio de la caza, y aun la administración de su hacienda; y llegó a tanto su curiosidad y desatino en esto, que vendió muchas hanegas de tierra de sembradura, para comprar libros de caballerías en que leer; y así llevó a su casa todos cuantos pudo haber dellos; y de todos ningunos le parecían tan bien como los que compuso el famoso Feliciano de Silva: porque la claridad de su prosa, y aquellas intrincadas razones suyas, le parecían de perlas; y más cuando llegaba a leer aquellos requiebros y cartas de desafío, donde en muchas partes hallaba escrito: 
\textit{la razón de la sinrazón que a mi razón se hace, de tal manera mi razón enflaquece, que con razón me quejo de la vuestra fermosura, y también cuando leía: los altos cielos que de vuestra divinidad divinamente con las estrellas se fortifican, y os hacen merecedora del merecimiento que merece la vuestra grandeza.}
Con estas y semejantes razones perdía el pobre caballero el juicio, y desvelábase por entenderlas, y desentrañarles el sentido, que no se lo sacara, ni las entendiera el mismo Aristóteles, si resucitara para sólo ello. No estaba muy bien con las heridas que don Belianis daba y recibía, porque se imaginaba que por grandes maestros que le hubiesen curado, no dejaría de tener el rostro y todo el cuerpo lleno de cicatrices y señales; pero con todo alababa en su autor aquel acabar su libro con la promesa de aquella inacabable aventura, y muchas veces le vino deseo de tomar la pluma, y darle fin al pie de la letra como allí se promete; y sin duda alguna lo hiciera, y aun saliera con ello, si otros mayores y continuos pensamientos no se lo estorbaran.

% Para una seccion entera:

\begin{quote}
\itshape
\bfseries
    la razón de la sinrazón que a mi razón se hace, de tal manera mi razón enflaquece, que con razón me quejo de la vuestra fermosura, y también cuando leía: los altos cielos que de vuestra divinidad divinamente con las estrellas se fortifican, y os hacen merecedora del merecimiento que merece la vuestra grandeza.
\end{quote}

% Mayor separacion.
\begin{flushleft}

A un olmo seco

Al olmo viejo, hendido por el rayo \\
y en su mitad podrido, \\
con las lluvias de abril y el sol de mayo \\
algunas hojas verdes le han salido. \\

%\smallskip
%\medskip
\bigskip

\textsc{Antonio Machado.}

\end{flushleft}

% Tipografias: https://tug.org/FontCatalogue/

% 5. MATEMÁTICAS: modo matemático --------------------------------------

% Para escribir formulas en la misma lineaz del texto usamos:ç
\section{Matemáticas}
kdfkdecndekjc $a²=b²+c²$  kdsjcndkjc

% Para poner formula centrada:
\[
a²=b²+c²
\]

% 6. BIBLIOGRAFÍA ------------------------------------------------------

% Interna: el entorno thebibliography. Tiene un parametro obligatorio que indica cuanto se tabula la referencia.
%\begin{thebibliography}{99}

    % El argumento de \bibitem es identificador unico para cada referencia.
    %\bibitem{Cer1585} La Galatea (1585)
    %\bibitem{Cer1605} El ingenioso hidalgo don Quijote de la Mancha
    %\bibitem{Cer1613} Novelas Ejemplares (1613)

    % Para crear referencias usamos comando \cite{etiqueta} (ver texto)

%\end{thebibliography}

% Externa: BibTeX (estilos)

\bibliographystyle{apalike} % plain, alpha, apalike
\bibliography{secciones/referencias.bib}

% Para elaborar archivo .bib, se toma el ISBN y se pega en una pagina isbn to bib

% 7. ALGUNOS PAQUETES ÚTILES -------------------------------------------
% Ver en preambulo e inicio del documento
% Hipervínculos en el PDF: hiperref
% Colores
% Columnas
% Hipervínculos

\end{document}
